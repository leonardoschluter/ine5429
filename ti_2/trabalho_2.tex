%-------------------------------------------------------------------------------
\documentclass[openany]{ufsctex/ufsctex}
%-------------------------------------------------------------------------------
% Packages

% Translation\usepackage{pdfpages} 

\usepackage[brazil]{babel}
\usepackage[utf8]{inputenc}
\usepackage{hyperref}
\usepackage{pdfpages}
\usepackage{listings}
\usepackage{longtable}

% General
\usepackage{color}  

 
\definecolor{codegreen}{rgb}{0,0.6,0}
\definecolor{codegray}{rgb}{0.5,0.5,0.5}
\definecolor{codepurple}{rgb}{0.58,0,0.82}
%-------------------------------------------------------------------------------
% User-commands
\newcommand{\todo}[1]{{\color{red}{#1}}}

%-------------------------------------------------------------------------------

\instituicao[a]{Universidade Federal de Santa Catarina} % Opcional
\departamento[a]{Departamento de Informática e Estatística}
\curso[o]{Ciências da Computação}
\documento[o]{Trabalho de conclusão de curso} % [o] para dissertação [a] para tese
\subtitulo{Trabalho Individual 2}
\titulo{Segurança em Computação}
%\subtitulo{Subtítulo (se houver)} % Opcional
\autor{Leonardo Schlüter Leite}
\grau{Bacharel em Ciências da Computação}
\local{Florianópolis} % Opcional (Florianópolis é o padrão)
\data{16}{abril}{2019}


\begin{document}
    \folhaderosto
    \sumario
    
    \chapter{Geradores de números pseudoaleatórios}
    Neste capítulo serão apresentados os dois algoritmos implementados para a geração de números pseudoaleatórios. Ambos os algoritmos selecionados produzem números múltiplos de 32, portanto números com 56 bits, 168, 224 e etc. não serão gerados.
    
    \section{Gerador Congruente Linear}
    Abaixo temos o código utilizado para gerar números aleatório usando LGC, baseando-se no algoritmo mostrado em https://aaronschlegel.me/linear-congruential-generator-r.html 
     \begin{lstlisting}[language=Java,breaklines=true, tabsize=2,basicstyle =\fontsize{9}{11}]
public class LinearCongruential {

    long modulus;
    long a; // multiplicador
    long c; // incremento
    long seed;

    // inicializacao do gerador
    // Para que o periodo da sequencia seja do tamanho 
    // do modulo eh necessario que o deslocamento ou
    // seja relativamente primo com o modulo
    // que a-1 seja divisivel por todos 
    // os fatores primos de m
    public LinearCongruential(long modulus, long a, 
    	long c, long seed){
    	
        this.modulus = modulus;
        this.a = a;
        this.c = c; // se c for = 0, caimos em outro 
        			//tipo de gerador.
        this.seed = seed;
    }

    public long generate(){
        this.seed = (a*this.seed +c) % modulus;
        // multiplica a semente, soma com constante 
        // e faz o modulo.
        // a seed eh atualizada e entao retornada.
        return seed; 
    }
}
\end{lstlisting}
	Tendo este código implementado, podemos percerber que os números gerados são de 64bit. Para gerar números que sejam múltiplos de 64, como 32, 128, 1024 e 4096 foi construído um loop que a cada iteração cria uma nova semente, inicializa um objeto do tipo LinearConruential e pede o próximo número. Então esses números são concatenados até termos a quantidade de bits necessária. Na tabela~\ref{tab:numerosLGC} temos exemplos de números de vários tamanhos gerados utilizando o algoritmo LGC.
	
\begin{longtable}{|p{3cm} | p{9cm} |}
 \caption{Exemplo de números gerados com LGC.}\label{tab:numerosLGC}\\
\hline
Tamanho em bits & Número Gerado  \\
\hline
31 & 1982274587 \\
\hline
63 & 7647551033894467049 \\
\hline
511 & 4487339197896165103959118044218800277 
4439806092253007016447113452822890646061608 
0851038371269032048288673820170285582936925 
9415897498595071394759598429613 \\
\hline
1023 & 812237707737715233392748949412645909792158
4156200002770059826252922922678883260357074016455
7416308904108916625008336311014371426803547486977
8315806651580409230750320778800842824037265645671
1313615119209623855698128160229770033050315635554
3835736811091460828166202201038954391496688421521
509392464365308518149 \\
\hline
2047 & 133875388389516398504794679886249058273696
3550505146432481406994129611692239266191110381182
9788109415713816627857207083012074316855387160589
8706811036192399219716020261922133532670520158054
4359048522073952207009051872071128440534844287149
0748686386866224474579717907067239357655113674851
2839884696560513677369673605173781255925654694451
6383060636108090887768503211080391927733196451434
3834228396222280458809625433748472226619983851991
3862510479680144492499378614729667166761507437305
4571115905226657988969455696562940111326545485325
538695556123492714498404690773376471228491925002
6987657231838100982214283648766262412\\
\hline
\end{longtable}


	 
    \section{Multiply-with-Carry}
        Abaixo temos o código utilizado para gerar números aleatório usando MWC, baseando-se no algoritmo mostrado em https://www.javamex.com/tutorials/random\_numbers/multiply\_with\_carry.shtml
    \begin{lstlisting}[language=Java,breaklines=true, tabsize=2,basicstyle =\fontsize{9}{11}]
public class MultiplyWIthCarry {

    final long a = 0xffffda61L;
    long seed = System.nanoTime() & 0xffffffffL;

    //Parecido com o LGC, porem passa em
    //alguns testes estatisticos que o LGC falha,
    //se forem escolhidos bons parametros.
    public int nextInt() {
        seed = (a * (seed & 0xffffffffL)) + (seed >>> 32); 
        // primeiro zera os 32 bits
        // mais significativo da semente,
        // depois multiplica e soma com a semenete
        // dividida por 2^32 com shift right.
        return Math.abs((int) seed);
    }

}
    \end{lstlisting}
    Tendo este código implementado, podemos percerber que os números gerados são de 32bit. Para gerar números que sejam múltiplos de 32, como 64, 128, 1024 e 4096 foi construído um loop que a cada iteração cria um novo objeto do tipo MultiplyWithCarry e pede o próximo número. Então esses números são concatenados até termos a quantidade de bits necessária. Abaixo temos alguns exemplos de números gerados com este algoritmo: 
 
\begin{longtable}{|p{3cm} | p{9cm} |}
 \caption{Exemplo de números gerados com MWC.}\label{tab:numerosMWC}\\
\hline
Tamanho em bits & Número Gerado  \\
\hline
31 & 1912360281 \\  
\hline
31 & 1697062396 \\
\hline
63 & 8081633200613964543 \\
\hline
63 & 7733642056664092968 \\
\hline
127 & 139146135588267251463073470093232054794 \\
\hline
127 & 142489858770018732778875980919996352105 \\  
\hline
255 & 34831164690570254176761523765974482045317
576655770892545132398369993994889285 \\
\hline
511 & 
64512565120576575476488228655115924471895509674723
38785114752729831791781378202144567519263387958490
37774611597961106016979593759252393
5187131540682925669 \\
\hline
1023 & 8432342347830620320615947194601875109659591659497
0552644938966090273130665389409798169320277031109
4597090074742525306093839703240434063817374330155
1160496229362402253946265710259922395533222700680
9935654348850142002119831311038224543451291024879
7317571809893797631884085882875442465747431544498
77490551364782 \\
\hline
2045 & 234770526248041897243529882238857384426575675682
3290223946978657547339232923569428804878073904038
5111600344987236414314259856384286244328268282262
3273164679719961176400321538832913884277954840130
6320760280984521312547086369237279770286895228148
1607574296478312355993381594201964491826303683628
4858913062917608434283318898496483642051374272849
6157107010057323859648205622533858612502392295284
7784286609915676863122899983873354655093704828457
5111086003311425874880159565583762686068197904802
2692115046027914387712221046447889928336556397368
7831561354207987182264589167098841737964046076219
8174542639509214971342685881 \\  
\hline
\end{longtable}

\section{Comparação}
	Segundo a ~\ref{tab:comparacaoMWCLGC}, podemos concluir que em média, o LGC executa em menos tempo utilizando as implementações apresentadas. Porém, é sabido que as implementações apresentadas podem não ser as mais eficientes, visto que a linguagem Java não é exatamente propícia para operações bitwise.

\begin{longtable}{| p{2cm} | p{2cm}| p{2cm}| p{4cm} |}
 \caption{Exemplo de números gerados com MWC.}\label{tab:comparacaoMWCLGC}\\
\hline
Tamanho em bits & Elapsed Time LGC & ElapsedTime MWC & LGC - MWC \\
\hline
32 & 2.41671E-4s & 3.77667E-4s & -1.3599598605651408E-4\\  
\hline
64 & 5.482E-6s & 5.161E-6s & 3.209997885278426E-7\\  
\hline
128 & 7.859E-6s & 7.045E-6s & 8.140000318235252E-7\\  
\hline
256 & 1.2502E-5s & 2.5166E-5s & -1.2664000678341836E-5\\  
\hline
512 & 2.0995E-5s & 2.1003E-5s & -7.999915396794677E-9\\  
\hline
1024 & 5.0945E-5s & 3.1665E-5s & 1.9279999833088368E-5\\  
\hline
2048 & 1.56663E-4s & 2.00992E-4s & -4.4329004595056176E-5\\  
\hline
4096 & 9.995E-5s & 1.27286E-4s & -2.733599831117317E-5\\  
\hline
8192 & 2.52915E-4s & 8.7043E-5s & 1.6587198479101062E-4\\  
\hline
16384 & 1.79728E-4s & 1.011E-4s & 7.86279997555539E-5\\  
\hline
\end{longtable}

\chapter{Números Primos}
	Neste capítulo será mostrado os dois métodos utilizados para verificar a primalidade de um número. Primeiro apresentaremos quatro números inteiros para cada potência de dois, começando com o expoente igual 5 e terminando em 12, gerados pelos algoritmos apresentados anteriormente. Depois será mostrado os códigos dos dois métodos verificadores, de forma comentada. Por último será apresentado uma comparação baseada em complexidade do algoritmo e nos tempos apresentados pela implementações propostas.
		

Dado esses números, agora será apresentados os métodos que serão utilizados para verificar se esses números são primos

\section{Miller-Rabin}
	Este método contém alguns contribuidores, sendo os principais Gary L. Miller e Michael O. Rabin. O primeiro conseguiu desenvolver um teste determinístico e o segundo propôs uma modificação para trazer probabilidade para dentro do verificador. Segue um exemplo de implementação: 
	 \begin{lstlisting}[language=Java,breaklines=true, tabsize=2,basicstyle =\fontsize{9}{11}]
public class MillerRabin {

    private int rounds;

    public MillerRabin(int numberOfRounds){
        this.rounds = numberOfRounds;
    }

    public boolean test(BigInteger number){
        // verifica se o numero e par.
        if (number.divideAndRemainder(BigInteger.TWO)[1]
                .equals(BigInteger.ZERO)) {
            return false;
        }
        //incrementador para a decomposicao do numero
        int powerOf2 = 0;
        // como e impar, diminuimos 1 para decompor
        BigInteger odd = number.subtract(BigInteger.ONE);

        //loop da decomposicao em potencia de 2
        //enquanto o mod por 2 for igual a 0
        while (odd.mod(BigInteger.TWO).equals(0)) {
            // atualiza o incrementador
            powerOf2++;
            //atualiza o valor para que o loop
            // tenha um fim ...
            odd = odd.divide(BigInteger.TWO);
        }



        Random random = new Random();
        for(int i = 0; i< this.rounds; i ++){
            BigInteger randomBase = selectRandomBase(number, random);
            // depois de ter selecionado a base, eleva ela
            // a potencia impar que sobrou da decomposicao
            // modulo o numero que se quer testar
            // este sera o divisor
            BigInteger divisor = randomBase.modPow(odd, number);

            //se divisor nao for igual a um e nao for igual ao numero - 2
            if( !divisor.equals(BigInteger.ONE) &&
                    !divisor.equals(number.subtract(BigInteger.TWO))){
               if(!this.teste(divisor, number, powerOf2)){
                   return false;
               }
            }
        }
        // se depois de todos os rounds
        // o metodo nao tiver retornado
        //o numero e um possivel primo.
        //se o numero nao for primo
        // o algoritmo tem 4^-(numero de rodadas) probabilidade
        // de dizer que o numero eh primo.
        return true;
    }

    private BigInteger selectRandomBase(BigInteger number, Random random){
        BigInteger randomBase;
        // seleciona uma base entre 2 e (n-2)
        do {
            randomBase = new BigInteger(number.subtract(BigInteger.TWO).bitLength(), random);
        } while (randomBase.compareTo(BigInteger.TWO) < 0 || randomBase.compareTo(number.subtract(BigInteger.TWO)) > 0);
        return randomBase;
    }

    private boolean teste(BigInteger divisor, BigInteger number,  int powerOf2){
        // verifica se o numero eh divisivel por
        // alguma potencia do divisior antes encontrado
        for(int j = 0; j < powerOf2; j++){
            divisor = divisor.modPow(BigInteger.TWO, number);
            // se for, retorna falso.
            if(!divisor.equals(number.subtract(BigInteger.ONE))){
                return false;
            }
        }
        return true;
    }
}

	 \end{lstlisting}
	
\section{Solovay–Strassen} 
	 Agora será apresentado uma implentação do algoritmo desenvolvido por  Robert M. Solovay e Volker Strassen. Apesar de um teste já superado por outros, como pelo próprio Miller-Rabin, este algoritmo foi escolhido pela importância ao demonstrar a possibilidade de por o algoritmo RSA em prática. Assim como o Miller-Rabin também é um teste probabílistico, sendo possível gerar a respsota errada. 
	 
	 \begin{lstlisting}[language=Java,breaklines=true, tabsize=2,basicstyle =\fontsize{9}{11}]
public class SolovayStrassen {
    private int rounds;
    public SolovayStrassen(int rounds){
        this.rounds = rounds;
    }

    public boolean test(BigInteger number){
        // verifica se o numero e par.
        if (number.divideAndRemainder(BigInteger.TWO)[1].equals(BigInteger.ZERO)) {
            return false;
        }
        Random random = new Random();

        for(int i =0; i < rounds; i++){
            BigInteger randomDividend = selectRandomDividend(number, random);
            // verifica o quociente da divisao
            BigInteger quotient = randomDividend.divide(number);

            //cria a potencia baseada em (numero - 1 ) /2 
            // que sera utilizada para testar se
            // o numero eh composto
            BigInteger power = number.subtract(BigInteger.ONE).divide(BigInteger.TWO);
            
            // se o quociente for igual a 0,
            // ou se o quociente mod  numero sendo testado
            // for igual ao dividendo elevado a potencia achada
            // anteriormente, entao o numero e composto 
            if(quotient.equals(BigInteger.ZERO) || quotient.mod(number).equals(randomDividend.modPow(power, number))){
                return false;
            }
        }
        return true;
    }

    private BigInteger selectRandomDividend(BigInteger number, Random random){
        // seleciona um dividendo aleatorio entre 2 e n - 1.
        BigInteger randomDividend;
        do {
            randomDividend = new BigInteger(number.subtract(BigInteger.ONE).bitLength(), random);
        } while (randomDividend.compareTo(BigInteger.TWO) < 0 || randomDividend.compareTo(number.subtract(BigInteger.ONE)) > 0);
        return randomDividend;
    }
    
    // nao podemos esquecer tambem que a probabilidade de erro 2^-(numero de rodadas)
}
\end{lstlisting}

\section{Comparação}
	Será mostrado tabelas com diferentes números de rodada para o algoritmo de Miller-Rabin, enquanto o do Solovay–Strassen deixaremos fixado o número de rodadas em 2000000000. Começaremos com 1000, depois 2000 e por último 20000. A primeira comparação antes de partirmos para dados reais é da complexidade. O Miller-Rabin quando mal implementado tem O(NumeroRodadas*log\^3(numeroSendoTestado)), mas podemos reduzir isso para O(NumeroRodadas*log\^2(numeroSendoTestado)). Enquanto o Solovay–Strassen tem complexidade de  O(NumeroRodadas*log\^3(numeroSendoTestado)). Desta forma podemos esperar que os tempos de execução, considerando que o Miller-Rabin não tenha sido acelerado ao máximo, serão iguais. Todos os tempos aqui listados são em segundos.
	
\begin{longtable}{|c|p{1cm}|p{1cm}|p{2cm}|p{2cm}|p{2cm}|}
 \caption{Tabela dos resultados e tempos dos algoritmos com 1000 rodadas. (MR = Miller-Rabin, SS = Solovay–Strassen, ET = Elapsed Time) }\label{tab:comparacao}\\
 \hline
 Índice & MR & SS & ET MR & ET SS & Diferença \\ \hline
 1 & Sim & Não & 0,02721946 & 0,00082451 & 0,02639494\\ \hline 
2 & Sim & Não & 0,00395581 & 0,00003180 & 0,00392401\\ \hline 
3 & Sim & Não & 0,00421093 & 0,00001616 & 0,00419477\\ \hline 
4 & Sim & Não & 0,00606996 & 0,00002150 & 0,00604846\\ \hline 
5 & Sim & Não & 0,01363707 & 0,00002138 & 0,01361570\\ \hline 
6 & Sim & Não & 0,00691495 & 0,00003078 & 0,00688417\\ \hline 
7 & Sim & Não & 0,00607304 & 0,00003843 & 0,00603461\\ \hline 
8 & Sim & Não & 0,00514969 & 0,00001684 & 0,00513284\\ \hline 
9 & Sim & Não & 0,01137630 & 0,00002545 & 0,01135086\\ \hline 
10 & Sim & Não & 0,01114735 & 0,00001635 & 0,01113100\\ \hline 
11 & Sim & Não & 0,01215473 & 0,00003235 & 0,01212238\\ \hline 
12 & Sim & Não & 0,01581122 & 0,00003540 & 0,01577581\\ \hline 
13 & Sim & Não & 0,03879687 & 0,00002254 & 0,03877433\\ \hline 
14 & Sim & Não & 0,05173407 & 0,00002865 & 0,05170542\\ \hline 
15 & Sim & Não & 0,04396776 & 0,00001802 & 0,04394974\\ \hline 
16 & Sim & Não & 0,02912524 & 0,00001844 & 0,02910680\\ \hline 
17 & Sim & Não & 0,11564312 & 0,00002016 & 0,11562296\\ \hline 
18 & Sim & Não & 0,10420472 & 0,00001519 & 0,10418953\\ \hline 
19 & Sim & Não & 0,11902215 & 0,00002068 & 0,11900147\\ \hline 
20 & Sim & Não & 0,13295890 & 0,00003048 & 0,13292842\\ \hline 
21 & Sim & Não & 0,54929798 & 0,00001912 & 0,54927887\\ \hline 
22 & Sim & Não & 0,49270492 & 0,00001969 & 0,49268523\\ \hline 
23 & Sim & Não & 1,05076475 & 0,00001630 & 1,05074846\\ \hline 
24 & Sim & Não & 1,06246156 & 0,00003153 & 1,06243002\\ \hline 
25 & Sim & Não & 4,86770054 & 0,00002803 & 4,86767251\\ \hline 
26 & Sim & Não & 4,78670085 & 0,00003337 & 4,78666748\\ \hline 
27 & Sim & Não & 6,13922313 & 0,00002661 & 6,13919652\\ \hline 
28 & Sim & Não & 5,04230766 & 0,00002859 & 5,04227907\\ \hline 
29 & Sim & Não & 39,54266595 & 0,00003313 & 39,54263282\\ \hline 
30 & Sim & Não & 38,74941144 & 0,00009253 & 38,74931891\\ \hline 
31 & Sim & Não & 37,17281331 & 0,00003463 & 37,17277868\\ \hline 
32 & Sim & Não & 42,53774093 & 0,00003393 & 42,53770700\\ \hline  
 \end{longtable}
 
 
 	
\begin{longtable}{|c|p{1cm}|p{1cm}|p{2cm}|p{2cm}|p{2cm}|}
 \caption{Tabela dos resultados e tempos dos algoritmos com 2000 rodadas. (MR = Miller-Rabin, SS = Solovay–Strassen, ET = Elapsed Time) }\label{tab:comparacao}\\
 \hline
 Índice & MR & SS & ET MR & ET SS & Diferença \\ \hline
 
1 & Sim & Não & ,07188966 & ,00358343 & ,06830623\\ \hline 
2 & Sim & Não & ,01092295 & ,00004786 & ,01087509\\ \hline 
3 & Sim & Não & ,00827426 & ,00001637 & ,00825789\\ \hline 
4 & Sim & Não & ,01022792 & ,00001961 & ,01020831\\ \hline 
5 & Sim & Não & ,01549784 & ,00015534 & ,01534250\\ \hline 
6 & Sim & Não & ,01036804 & ,00002936 & ,01033868\\ \hline 
7 & Sim & Não & ,00992531 & ,00002481 & ,00990050\\ \hline 
8 & Sim & Não & ,00995677 & ,00003534 & ,00992143\\ \hline 
9 & Sim & Não & ,02164093 & ,00001231 & ,02162862\\ \hline 
10 & Sim & Não & ,01794020 & ,00002112 & ,01791908\\ \hline 
11 & Sim & Não & ,01740474 & ,00001803 & ,01738670\\ \hline 
12 & Sim & Não & ,01651986 & ,00001889 & ,01650097\\ \hline 
13 & Sim & Não & ,05196184 & ,00001506 & ,05194679\\ \hline 
14 & Sim & Não & ,05953822 & ,00002791 & ,05951031\\ \hline 
15 & Sim & Não & ,04452142 & ,00001748 & ,04450393\\ \hline 
16 & Sim & Não & ,05643745 & ,00001414 & ,05642330\\ \hline 
17 & Sim & Não & ,19155928 & ,00001875 & ,19154053\\ \hline 
18 & Sim & Não & ,17732421 & ,00001681 & ,17730740\\ \hline 
19 & Sim & Não & ,17727300 & ,00002178 & ,17725122\\ \hline 
20 & Sim & Não & ,18070891 & ,00002645 & ,18068246\\ \hline 
21 & Sim & Não & 2,18059601 & ,00001692 & 2,18057909\\ \hline 
22 & Sim & Não & 1,84947173 & ,00003092 & 1,84944081\\ \hline 
23 & Sim & Não & 1,11954804 & ,00002089 & 1,11952716\\ \hline 
24 & Sim & Não & 1,52860683 & ,00001426 & 1,52859256\\ \hline 
25 & Sim & Não & 11,01797639 & ,00004364 & 11,01793276\\ \hline 
26 & Sim & Não & 11,19984695 & ,00004088 & 11,19980607\\ \hline 
27 & Sim & Não & 10,98346072 & ,00002964 & 10,98343108\\ \hline 
28 & Sim & Não & 12,17771666 & ,00003624 & 12,17768042\\ \hline 
29 & Sim & Não & 87,54197128 & ,00003224 & 87,54193904\\ \hline 
30 & Sim & Não & 89,05598291 & ,00005919 & 89,05592372\\ \hline 
31 & Sim & Não & 82,28403497 & ,00002042 & 82,28401455\\ \hline 
32 & Sim & Não & 82,27667705 & ,00007727 & 82,27659978\\ \hline 
 
 \end{longtable}
	
	
 
 	
\begin{longtable}{|c|p{1cm}|p{1cm}|p{2cm}|p{2cm}|p{2cm}|}
 \caption{Tabela dos resultados e tempos dos algoritmos com 20000 rodadas. (MR = Miller-Rabin, SS = Solovay–Strassen, ET = Elapsed Time) }\label{tab:comparacao}\\
 \hline
 Índice & MR & SS & ET MR & ET SS & Diferença \\ \hline
 
1 & Sim & Não & 0,11171003 & 0,00072251 & 0,11098752\\ \hline 
2 & Sim & Não & 0,07335955 & 0,00011614 & 0,07324341\\ \hline 
3 & Sim & Não & 0,08186430 & 0,00001901 & 0,08184529\\ \hline 
4 & Sim & Não & 0,05165693 & 0,00001826 & 0,05163867\\ \hline 
5 & Sim & Não & 0,13971304 & 0,00012893 & 0,13958411\\ \hline 
6 & Sim & Não & 0,13462999 & 0,00003382 & 0,13459617\\ \hline 
7 & Sim & Não & 0,11232166 & 0,00002404 & 0,11229762\\ \hline 
8 & Sim & Não & 0,11554731 & 0,00002949 & 0,11551782\\ \hline 
9 & Sim & Não & 0,24554998 & 0,00003056 & 0,24551941\\ \hline 
10 & Sim & Não & 0,22760821 & 0,00004261 & 0,22756561\\ \hline 
11 & Sim & Não & 0,19712433 & 0,00003155 & 0,19709278\\ \hline 
12 & Sim & Não & 0,20150501 & 0,00003131 & 0,20147369\\ \hline 
13 & Sim & Não & 0,72246317 & 0,00003407 & 0,72242911\\ \hline 
14 & Sim & Não & 0,73429265 & 0,00003507 & 0,73425758\\ \hline 
15 & Sim & Não & 0,71114959 & 0,00003585 & 0,71111373\\ \hline 
16 & Sim & Não & 0,43449985 & 0,00004040 & 0,43445945\\ \hline 
17 & Sim & Não & 2,46091771 & 0,00002405 & 2,46089366\\ \hline 
18 & Sim & Não & 2,56907771 & 0,00004308 & 2,56903463\\ \hline 
19 & Sim & Não & 1,63278166 & 0,00002478 & 1,63275688\\ \hline 
20 & Sim & Não & 3,39448872 & 0,00007026 & 3,39441846\\ \hline 
21 & Sim & Não & 14,56426880 & 0,00003966 & 14,56422914\\ \hline 
22 & Sim & Não & 14,36132480 & 0,00007242 & 14,36125237\\ \hline 
23 & Sim & Não & 15,49475016 & 0,00004771 & 15,49470245\\ \hline 
24 & Sim & Não & 16,97889106 & 0,00001516 & 16,97887590\\ \hline 
25 & Sim & Não & 112,70947719 & 0,00002850 & 112,70944869\\ \hline 
26 & Sim & Não & 119,64860039 & 0,00001713 & 119,64858326\\ \hline 
27 & Sim & Não & 117,71940390 & 0,00002893 & 117,71937497\\ \hline 
28 & Sim & Não & 116,97791633 & 0,00002738 & 116,97788895\\ \hline 
29 & Sim & Não & 903,69619122 & 0,00002027 & 903,69617095\\ \hline 
30 & Sim & Não & 849,73224558 & 0,00007467 & 849,73217091\\ \hline 
31 & Sim & Não & 803,39407188 & 0,00002740 & 803,39404448\\ \hline 
32 & Sim & Não & 858,20336123 & 0,00001957 & 858,20334166\\ \hline 

 \end{longtable}
	
	
	Nas tabelas acima temos dois comportamentos inesperados. O primeiro é que enquanto o Solovay-Strassen tem sua duração muito baixa, o Miller-Rabin tem seu tempo de execução mais que dobrando quando alteramos a quantidade de bits para o dobro. Motivos para isso podem ser vários, mas o mais óbvio é a escolha da linguagem. Sabemos que Java não é a linguagem mais robusta, muito menos seu ambiente de execução. Outro possível problema é a dependência que a implementação tem da classe BigInteger para lidar com números exageradamente grandes. O Outro comportamento é que com 1000, 2000 e 20000 rodadas, o Miller-Rabin não foi capaz de identificar que os números não eram primos. Enquanto o Solovay-Strassen identificou muito fácil. Obviamente isso acontece pois o número de rodadas é muito diferente entre as duas implementações. Porém, isto é uma barreira de hardware/software disponíveis e escolhidos para o desenvolvimento deste trabalho. Portanto, de acordo com todos os códigos apresentados e dados listados, podemos concluir que neste caso específico a implementação do Solovay-Strassen apresentou melhores resultados.
	
	
	 
\begin{longtable}{|c|p{9cm}|}
 \caption{Tabela de números aleatórios primos.}\label{tab:comparacao}\\
  \hline
   	Índice & Número \\ \hline
   	1 & 2029663043\\ \hline 
2 & 1818701935\\ \hline 
3 & 1269497125\\ \hline 
4 & 2066721673\\ \hline 
5 & 7630766861971458425\\ \hline 
6 & 8305496549126277389\\ \hline 
7 & 7622826425035437111\\ \hline 
8 & 7040092047322450767\\ \hline 
9 & 120298170997570680356444293518225934565\\ \hline 
10 & 133804528563206615538729272032743639375\\ \hline 
11 & 162811172248064604546734121441704956247\\ \hline 
12 & 125680834290773339795990496296686958291\\ \hline 
13 & 53215508396768921864158266896585268430336999318
646838633790285185186997324917\\ \hline 
14 & 40533387024726302616725978853770682440451470247
015357489371785044403544074253\\ \hline 
15 & 50502215623936459773444886938277764357408507917
167559671340915625223036862831\\ \hline 
16 & 37734798666453584805041211019807942609813608663
655818417264849597749727573747\\ \hline 
17 & 
36784024343224613212559066245057212238014342749
80623156784568725896021848104778576704045031428
32774527703592792175178308492505461418621570750
3034566100539\\ \hline 
18 & 
56825781112553700490447430168742749383412535001
26556254774860280317538523703915964539415705371
70443377480260544128892227592764591285388238177
9991232856009\\ \hline 
19 & 
37528720125794310288937004851364243958326795809
40651767609704806360556501049520147660521852605
87827457428974335715570652403320475451066954544
3375434319783\\ \hline 
20 & 
56971371811348059825093807875808982782138683074
87270084484693844529759960002955606047402213272
03950408186257729813384481076318081250684317120
7462213122379\\ \hline 
21 & 
67070175241581097513176190774351287166953940781
62212134354146409421311659777730953400907447351
72687809209125622836612169773670937086932924925
32445454197623366508556427506681133327621266758
04618810400825102026210689501802632981977324424
97233857375390305733502475950890675886079368610
34388423329835128233056113\\ \hline 
22 & 
76029136412596490979197160539457113279797433033
42666950244411032008654439987380268689081295278
23832447840127253328190292216663380345595100757
25264237787170471930437119969684211278159790513
71008395727018531608545070911117162639728027573
45608221251697917759445569694683965497018484031
17689231729587590958853333\\ \hline 
23 & 
66134819633397037237609151550203988817500456486
95720427323914180668691926641401420467448165505
02314580421162016163678279885948160004851867176
37437196612925867473633333070627040038856390783
41979985023912914826977462340578280347766366312
31365877675311809321485256691730643635961661416
91428707948138486244956699\\ \hline 
24 & 
79262019616818438601650147906191545917783951859
01264539785282623296805597102398658551251672065
29253122979458871585714980826441528674239985510
50429360726310272275013427247238161272861282596
76277035930433739590111243976706641806609979008
50936208871648970265765134270907407636709269867
82562324682727591311010025\\ \hline 
25 & 
13371479341546463597378969830319908828669466900
14354342028546467597392946333122207573261650192
20550976142683097445604595702857143515171188507
15211053313804215641516978017903523115552105333
88963575524376554025706606694415355921163149496
16090859174303030099023036635748092221071968790
95402376421822352087438902430199549737553256781
40795243898693035036247612424071444841676805585
65484243691046207604500881039257841371097774381
49903077389774552300245146925826218561023299575
03918368107401497250173705179112147242625582195
03542441202221931275745578344373455959833468168
69761507258627677187584157599924613509165153870
617873\\ \hline 
26 & 
12503047789900648976806157045988977119866494414
97103401985979203794386771803486562987187870806
56030736192291694797884430332448907700525510617
61330210163788706585160635955865834843406487159
44607702173153902528758973128937280487879490941
47680811802458571320235377238196229132587140859
25297897865652854364716243728609923219886470797
50372095051177334033412668309783646378377538417
19956383174070697347356315584807485043992021947
60283236778490305559338123809012886350318839484
44396462536338948016777188831640886768112528915
59040633659750487450378578483245203406039145458
29704876891793398983769257344275483559267937343
255453\\ \hline 
27 & 
13357722463604316822300176461233422623216434475
24276050119455105317682471052144634745693890799
65551830170706843277456313365786453354117205258
45495347803789805275338192012683280133479505670
46258703053816798492254447955332653685861302317
01397916210271422236765210362212000987656081473
54681152931061765204327103009148722745343698410
28939542228123953037814048611433180759916933746
56774632736212339796286553093727160184402404635
85154973556121986150979524223731276225404497982
79775777808684584392135886595602752968984639513
30939244576098594053461710263510887166990341174
92472263689698192672115748683130797488012972909
271545\\ \hline 
28 & 
12526360337564578028267997795111411862108433975
96252436664657196591100320680444037794098412806
12417997059565106342490708151211091425432792214
94773804674921341435990555691878644569268571938
77536598894215747143678045278285507353191672941
40541298146286208242970111851116051679628003840
52849924244203531702254263344105092249867100333
29148505939765060670684674554929621550045522408
87576545284997654403377905307500372363602151338
93109297709105709696754988562196477396309180155
03224808666724756673014078288734522426603546033
13264096361823071130104961098380564536389640702
29057552750559893020145294078174152913439078051
578095\\ \hline 
29 & 
4997467479754343465478030653202740512179905548
4073138168826389669280266085411190241190240306
1984709585390010574444897735366506055508830298
1335651857083921318642673073029198730819262308
7998675335723995557319690514203748490419988701
6796361958633456676891796594803861894652314867
8849842857603014373134525299794078691886854810
5603908015423664977543898648278786567743226997
5416664369348082878785110325403656214628123657
8343981594714934255035240916121247318527972388
2201234636702482703811610985198797601698765040
3508460895380636459400158418849121590902661338
5748853315369723214248872461902419416573439542
2200897639482130324346092642743868957401067686
7013116433046606838088959209120832514493173015
7312369868544224841184492941539656908033003270
69043964253760984321027021178046511033251334870
24320963295387572001417776428516000839066484952
82348977075058852574848543950838806946030762212
26853022104484680321940822704091035051122519715
12738481129282130490855126442787777504646995883
11153186979704230151680779643240124809252419805
82120370426231115163816530205520468835479316234
70338418403100633871651476961221162337212504760
48642807459290789926155699342524429094135948217
73496553678500305809311377744455041539012410776
120868859699255688045166647\\ \hline 
30 & 
43211492634276783754495948928515095427988834473
23042067703695440993936885154733278587253309719
62099240899366022233574589952236588965669362843
26715113226229563506000928322912682994666465121
87696383129164659704880708247464555612715431404
15630215043676943245598118762907208336879339052
21009592766753348890688385332666351027637550429
68030860348288306315095594109582777823601830290
62792845278293529267816585437310868449287878891
19321321396151597693485380238906786332559337153
42308799049100640333934318281421731353558538919
14438227759932085414301988924045664298678306156
61209525924618410551012613446247349871361880743
36616811460565014236632093444146624921164025283
48449131687869447004895777579260792741131733963
48886339177056078982926392440587830093869435641
13046907740390146107083855093754734846339634217
93704283539257238171026902372961631009108740918
54506485187141058361924917862312981308758819806
31429976313247611839815191043311597612728252946
66519454203222490986401509182133072708018099358
80411112875712611979989180994550562466865024239
39037132930393987657484948152291278353415278706
82368690712262905241408522816946170377352071022
43147211794005736693342189222886804912980036687
31953833497307941370522364693099839753174419531
63701294607\\ \hline 
31 & 
49984710841722523575153694013222036663872958578
83741250679222263998479122216653674064413863160
19919749008786504188137495104771704570044258171
78190633494240559189500320955071222981995834863
89936395549870324994517461824314307479236067878
27943682450646195931806709793459545427861446166
67358148620434375319200142754813103960214923103
41481619573994720321446671673394824840510083494
18379225637301211540009993348048857914918463104
41301160845346100702375044527571124420318647741
71669844097433466208150225806390676659280704168
31401334320102962864103802615150600780595750928
42077053834984148832663978320889032203952297564
06897743074208271580428497443631162956224951585
93440420082842912954754329639410292075422563742
79713428872424850335548012251172695444255450181
34588221080130925821396063965977044112770917733
37878554082698932329124609978637258410253828247
89184058687857243884604552679218651933487269182
81302020843029865690978802141373182139512877601
05260566305668886701751572123123030744229596311
08592864939704918866062164981920934133596447152
72841383904137959606350882245353231823102249292
15435158068506580679094760737149237515791675172
99582012312261294644176802504600473092985255024
08793555301976007738999775092702235499789892748
20978348327\\ \hline 
32 & 
37274233314568899632336507339050019789629231496
06062029686083159997299830441140288599528403812
22398523702755637336721939650587594947948846083
89545255103822833685482527687289578622423435612
86416635699223738998005754188389053446753360218
31464642319789751403194208976586070433110370385
05592618568036443061992519519719009338344589830
44488437234284711651809097435974075241967365634
72146132762651519392059492481647581696620455398
03875420544179045671080895733802929115871249179
62133770885709183660229915987453981495137220495
88646682012126984835436923154339859963060749253
01594390839263701448765233736864102207106814661
04350615707034766992674453524390710870819951506
01852181089675322885031475237424423825157621493
36965736887149132379504908301827132018087144070
32062145629308174225545179785250428483096090946
34861385077783117198030672816309307049699133097
82952695273067585613131052909138679031105577400
37840163001329011139289305521862477012725358782
29195775060579354446808334159421474542048448365
85752376432650430314321084391645478441757689635
95714855588446207564456047049009326015530030953
61868904046835396305151371360444659458136647671
90528787131934564299965488828855213611220870752
87432303843976070544079032825364542972488144223
35496429885\\ \hline 

\end{longtable}

\end{document}